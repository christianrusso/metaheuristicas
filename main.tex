\documentclass[11pt,a4paper]{article}

\usepackage{geometry}
\usepackage[spanish, activeacute]{babel}
\usepackage[utf8]{inputenc}
\usepackage{amsthm}
\usepackage{amsmath}
\usepackage{xfrac}
\usepackage{amsfonts}
\usepackage{amssymb}
\usepackage{graphicx}		% \includegraphics
\usepackage{float}			% Para fijar figuras y tablas exactamente donde uno quiere.
\usepackage{color}
\usepackage{clrscode3e}		% Algoritmos tipo CLRS.
\usepackage{caratula} 		% Utilitarios para armar la caratula.
\usepackage{euler}			% Texto matematico estilo Concrete Mathematics
\usepackage{url}            % Para hacer hipervínculos

\usepackage{footnote}
\makesavenoteenv{tabular}	% Para poner la footnote del capo del sorting.

\setcounter{secnumdepth}{5}

\renewcommand{\rmdefault}{pplx}	% Fuente estilo Concrete Mathematics.

% Codigo fuente.
\usepackage{listings}
\lstset{
	language=C++,
	basicstyle=\small\sffamily,
	numbers=left,
	numberstyle=\tiny,
	frame=tb,
	columns=fullflexible,
	showstringspaces=false
}

\newcommand\todo[1]{\Large\textbf{\textcolor{red}{#1}}\normalsize}


\usepackage{chngcntr}	% Numeracion granular de distintos entornos.
\counterwithin{table}{section}
\AtBeginDocument{\counterwithin{lstlisting}{section}}


% Tablas de archivos de test.
\usepackage{verbatim}	% \verbatiminput
\newcommand{\inoutsamplesfile}[3]{
     %\vspace{1\baselineskip}
     \begin{table}[H]
     %\newcommand{\testdir}{tests}
     \renewcommand{\tablename}{Test}
     \caption{\texttt{#3}}
     \begin{center}
     \begin{tabular}{|l|l|} \hline
          \textbf{Entrada} & \textbf{Salida} \\ \hline
          \begin{minipage}[t]{0.45\textwidth} \verbatiminput{#1} \vspace{-2ex} \end{minipage} & \begin{minipage}[t]{0.45\textwidth} \verbatiminput{#2} \vspace{-2ex} \end{minipage} \\ \hline
     \end{tabular}
     \end{center}
     \end{table}
}


\begin{document}

\parskip=5pt

\thispagestyle{empty}

% Caratula.
\def\Materia{Metaheuristicas}
\def\Titulo{Trabajo Pr'actico}
\def\Fecha{2 de septiembre de 2016}

\begin{center}
    {\LARGE\textbf{\Materia}}\\[1em]    
    \vspace{5mm}
    {\Large \textbf{\Titulo}}\\[1em]
    \vspace{2mm}
    {\textbf{\large \Fecha}}\\
    \vspace{5mm}
    \textbf{\tablaints}
\end{center}

\tableofcontents

\newpage

\pagestyle{headings}
\setcounter{page}{1}

%\input{ejemplos}
\newpage
\section{Introducción Teórica}\label{sec:introduccion}
\subsection{Colonia de Hormigas}
Es una metaheuristica de la familia de PSO (Particle Swarm Optimization) basada en el comportamiento en grupo de las hormigas para definir el camino a un recurso deseado, en otras palabras es una metodología inspirada en el comportamiento colectivo de las hormigas en su búsqueda de alimentos. 
Es muy usada para solucionar problemas computacionales que pueden reducirse a buscar los mejores caminos o rutas en grafos. Es por eso que es muy importante recordar que las hormigas son prácticamente ciegas, y sin embargo, moviéndose al azar, acaban encontrando el camino más corto desde su nido hasta la fuente de alimentos (y regresar).
Entre sus principales características se encuentran:

\begin{enumerate}
\item Una sola hormiga no es capaz de realizar todo el trabajo sino que termina siendo el resultado de muchas hormigas en conjunto.
\item Una hormiga, cuando se mueve, deja una señal química en el suelo, depositando una sustancia denominada \textbf{feromona}, para que las demás puedan seguirla.
\end{enumerate}

De esta forma, aunque una hormiga aislada se mueva esencialmente al azar, las siguientes decidirán sus movimientos considerando seguir con mayor frecuencia el camino con mayor cantidad de feromonas.

La metaheurística general consiste en lo siguiente:
\begin{enumerate}
\item En principio, todas las hormigas se mueven de manera aleatoria, buscando por si solas un camino al recurso que estan buscando (una posible solución).
\item Una vez encontrada una solucion, la hormiga vuelve, dejando un rastro de feromonas; este rastro puede ser mayor o menor dependiendo de lo buena que sea la solución encontrada. 
\item Utilizando este rastro de feromonas, las hormigas pueden compartir información entre sus distintos pares en la colonia.
\item Cuando una nueva hormiga inicia su trabajo, es influenciada por la feromona depositada por las hormigas anteriores, y así, aumenta las probabilidades de que esta siga los pasos de sus anteriores
al acercarse a un recurso previamente encontrado.
\end{enumerate}


\begin{figure}[h]
\centering
\caption{Ejemplo convergencia a una solución}
\includegraphics[width=5cm]{imagenes/feromona}
\end{figure}

\begin{figure}[h]
\caption{Ejemplo de uso de feromona}
\centering
\includegraphics[width=5cm]{imagenes/feromona2}
\end{figure}

En la \textbf{figura 1}, podemos ver una serie de iteraciones donde las hormigas llegan a la Fuente de comida y vuelven, dejando feromonas y en la siguiente iteración la solución se ve influenciada por la feromona. 

Finalmente se llega a una camino, el cual es elegido por casi todas las hormigas, siendo este la soluci\'on final.

En la \textbf{figura 2}, asumiendo que el número de lineas punteadas es proporcional a la cantidad de feromona, se puede ver como el camino inferior es más corto que el superior, por lo cual muchas más hormigas transitarán por éste durante el mismo per\'iodo de tiempo. Esto implica, que en el camino más corto se acumula más feromona mucho más rápido. Después de cierto tiempo, la diferencia en la cantidad de feromona en los dos caminos es lo suficientemente grande para influenciar la decisión de las nuevas hormigas que entren a recorrer estas vías

Se puede ver que una gran ventaja de esta metaheurística es que puede construir una solución intercambiando información entre las distintas hormigas (soluciones), así generar una solución mejor de la que podrían generar individualmente.

Con el paso del tiempo, el rastro de feromonas comienza a evaporarse, y esto produce que los caminos pierdan su fuerza de atracción, cuanto más largo sea el camino, más tiempo demorara una hormiga en recorrerlo, más se evaporará la feromona y por ende serán menos frecuentado. Por su parte los caminos mas cortos (o mas óptimos) tendrán mayor cantidad de feromonas, por ende, mayor probabilidad de ser frecuentados.

ACO fue el primer algoritmo de optimizacion de Colonias de Hormigas desarrollado por Marco Dorigo en su tesis doctoral \cite{paperDorigo}. 

Algunas de las aplicaciones donde se utiliza esta metaheurística:
\begin{enumerate}
\item El problema del viajante de comercio (TSP)
\item Optimización para el diseño de circuitos lógicos combinatorios
\item Problemas de enrutamiento de vehículos
\item Problema de la asignación de horarios
\item Aplicaciones a análisis de ADN y a procesos de producción
\item Partición de un grafo en árboles:
\item Otros
\end{enumerate}



\begin{thebibliography}{9}
\bibitem{wikipedia} 
https://en.wikipedia.org/wiki/Ant\_colony\_optimization\_algorithms

\bibitem{claseMetah} 
http://www-2.dc.uba.ar/materias/metah/meta2016-clase7.pdf

 
\bibitem{paperDorigo} 
http://people.idsia.ch/~gianni/Papers/CEC99.pdf

\bibitem{paperAplicaciones} 
Ant colony optimization: applications and trends. Carlos Algarin

\end{thebibliography}
\newpage
\section{El problema}\label{sec:problema}

Un yacimiento petrol\'\i fero es una acumulaci\'on natural de hidrocarburos (gas natural y petr\'oleo, entre otros) en el subsuelo. Debido a la creciente escasez de reservas de hidrocarburos acumulados en yacimientos convencionales, la industria del petr\'oleo y diversos gobiernos nacionales han tornado su atenci\'on en las \'ultimas d\'ecadas a la explotaci\'on de yacimientos no convencionales. Uno de los tipos de yacimientos m\'as explorados est\'a dado por las reservas de petr\'oleo y gas natural almacenados en un tipo de rocas sedimentarias llamadas pelitas (shale), conocidos como yacimientos de \emph{shale gas} y \emph{shale oil}.
 
La explotaci\'on de este tipo de yacimientos utiliza m\'etodos de fractura hidr\'aulica, por medio de los cuales se generan fracturas en la roca madre para concentrar el petr\'oleo y el gas natural y posteriormente proceder a su extracci\'on. A pesar de que las primeras inyecciones de material para la extracci\'on de hidrocarburos se remontan a la segunda mitad del siglo XIX, reci\'en se comenz\'o a usar este tipo de m\'etodos en forma extensiva a prin\-ci\-pios del siglo XXI, principalmente en Estados Unidos. Adem\'as de las reservas en Estados Unidos, en la \'ultima d\'ecada se han descubierto enormes reservas de shale gas y shale oil en Argentina, Canad\'a y China.

Se describe el proceso de explotaci\'on de un yacimiento \emph{shale}. En primer lugar, se realizan varias perforaciones verticales en el subsuelo que llegan hasta la roca madre. Como se ve a continuaci\'on:

\begin{center}
\includegraphics[width=0.6\textwidth]{imagenes/figura1}
\end{center}

El sector en la superficie alrededor de las bocas de pozo se denomina locaci\'on, y habitualmente ocupa un \'area rectangular de entre algunas decenas y unos pocos cientos de metros por lado. Estos equipos son los \'unicos que se ven en la superficie, y habitualmente su instalaci\'on involucra obras de nivelaci\'on del suelo y construcci\'on de caminos de acceso. Como consecuencia, las locaciones no pueden estar sobre cursos de agua, barrancos o en sitios montanosos.

\begin{center}
\includegraphics[width=0.45\textwidth]{imagenes/figura2}
\includegraphics[width=0.45\textwidth]{imagenes/figura3}
\end{center}

Cada perforaci\'on atraviesa la roca madre, y a lo largo de esta perforaci\'on
se realizan los procesos de inyecci\'on de materiales para lograr la fractura de
la roca. Luego, se utilizan las mismas para la extracci\'on de los hidrocarburos
que migran hacia las zonas de fractura.

La zona
explotada a partir de una locaci\'on se denomina pad, y tiene una forma t\'ipicamente
rectangular.

Dadas estas caracter\'isticas del problema queremos que las zonas de fractura en la roca madre no se deban superponer.


Al momento de planificar la explotaci\'on de un yacimiento no convencional,
uno de los principales problemas a resolver es donde ubicar las locaciones y
que tipo de explotaci\'on realizar en cada una (lo cual determina el tipo y
tamano de los pads resultantes) con el objetivo de maximizar la producci\'on
y minimizar los costos y el impacto ambiental. Este problema se conoce con
el nombre de optimizaci\'on del \'area de drenaje, y como resultado se espera un
plan de explotaci\'on que muestre las ubicaciones de locaciones y pads.


En la siguiente figura podemos ver  el mapa de un yacimiento, y las configuraciones de pads que
podemos usar para la explotaci\'on.

\begin{center}
\includegraphics[width=0.6\textwidth]{imagenes/figura4}
\end{center}

Los pads se deben ubicar siguiendo cierto \'angulo $\alpha$, llamado direcci\'on de esfuerzo horizontal m\'inimo.
Como por ejemplo:

\begin{center}
\includegraphics[width=0.45\textwidth]{imagenes/figura5}
\includegraphics[width=0.45\textwidth]{imagenes/figura6}
\end{center}

Formalmente, los datos de entrada del problema est\'an dados por los siguientes
elementos:

\begin{enumerate}
\item El yacimiento Y  $\subseteq$ $\mathbb{R}^2$, que en este trabajo asumimos dado por un pol\'igono en el plano (no necesariamente convexo). Todos los pads deben estar ubicados dentro del per\'imetro del yacimiento.
\item Una funci\'on ogip : Y $\rightarrow \mathbb{R}_{+}$ (original gas in place), que especifica la
cantidad de shale gas esperada en cada punto del yacimiento, y el precio
de venta $\rho  \in \mathbb{R}_{+}$ de cada unidad extra\'ida. 

Dado un pad P $\subseteq$ Y ubicado
dentro del yacimiento, el gas total obtenido por explotar el pad esta dado
por ogip(P) := $\int_{P}^{} ogip(x) dx$
\item Un conjunto S = \{$S_1$,...,$S_k$\} de configuraciones posibles de pads, que
podemos utilizar para explotar el yacimiento. Para cada configuraci\'on S $\in$ S, tenemos estos datos:
\begin{enumerate}
\item  Largo $lp_S \in \mathbb{R}_{+}$ y ancho $ap_S \in \mathbb{R}_{+}$ del pad, en metros.
\item  Largo $ll_S \in \mathbb{R}_{+}$ y ancho $al_S \in \mathbb{R}_{+}$ de la locaci\'on en metros, y asumimos que $ll_S < lp_S$ y $al_S < ap_S$.
\item  La locaci\'on esta ubicada en el centro del pad, pero se puede mover algunos metros de este centro para evitar obst\'aculos geogr\'aficos. El par\'ametro de tolerancia $tol_S \in \mathbb{R}_{+}$ especifica la cantidad m\'axima de metros que el centro de la locaci\'on se puede mover con relaci\'on
al centro del pad.
\item  Finalmente, se tiene el costo $c_S \in \mathbb{R}_{+}$ de construcci\'on del pad.
Dado un pad P correspondiente a la configuraci\'on S, definimos su margen neto como $neto(P) := \rho$ X $ogip(P) - c_S$.
\end{enumerate}
\item  Un conjunto de obst\'aculos (habitualmente de \'indole geogr\'afica) que las locaciones deben evitar. Consideramos que cada obst\'aculo esta dado por un pol\'igono en el plano, y ninguna locaci\'on se puede superponer con ning\'un obst\'aculo.
\item Un \'angulo $\alpha \in [0; 2\pi]$ de explotaci\'on ideal, denominado \'angulo de esfuerzo horizontal m\'inimo, que especifica la orientaci\'on aproximada que deben tener los pozos horizontales sobre el yacimiento con relaci\'on al norte geogr\'afico. Como esta orientaci\'on es aproximada, se tiene una tolerancia $\beta \in [0; 2\pi]$, que especifica que todos los pads deben estar orientados en un \'angulo comprendido en el intervalo $[\alpha - \beta, \alpha + \beta]$.
\end{enumerate}



El problema consiste en hallar un conjunto de pads Ps = \{$P_1, ... ,P_n$\} y un conjunto de locaciones Ls = \{$L_1, ..., L_n$\} (de modo tal que la locaci\'on $L_i$ corresponde al pad $P_i$, para i = 1,...,n) que maximice $neto(P) :=  \sum_{i=1}^{n} neto(P_i) $ de modo tal que se cumplan las siguientes restricciones:

\begin{enumerate}
\item Todos los pads deben estar contenidos dentro del yacimiento, es decir $P_i \subseteq Y$ para i = 1,...,n.
\item Como restricci\'on, los pads de la soluci\'on no se deben superponer, dado que corresponden a zonas de fractura en la roca madre.
\item Cada pad y su locaci\'on deben responder a las especificaciones de una conguraci\'on de S. Es decir, para cada i = 1,...,n debe existir una configuraci\'on S $\in$ S tal que $P_i$ tiene largo $lp_S$ y ancho $ap_S$, $L_i$ tiene largo $ll_S$ y ancho $al_S$ y su centro esta a no m\'as de $tol_S$ metros del centro de $P_i$, y finalmente $P_i$ y $L_i$ est\'an orientados en un mismo \'angulo, el cual debe estar entre $[\alpha - \beta, \alpha + \beta]$.
\item Ninguna locaci\'on de Ls se debe superponer con ning\'un obst\'aculo.
\end{enumerate}


Por ejemplo, en la siguiente figura se muestra un yacimiento y los obst\'aculos dentro del yacimiento, y en la figura contigua se muestra una soluci\'on factible para $\alpha = \pi / 4$ y para dos configuraciones posibles. Dado que la funci\'on ogip no siempre esta bien determinada de antemano (y en ocasiones se trabaja con estimaciones poco fiables de esta funci\'on) alternativamente se puede solicitar que se maximice el \'area total cubierta con los pads propuestos, en lugar del beneficio neto total obtenido. El algoritmo que se presenta en la pr\'oxima secci\'on permite utilizar cualquiera de estas dos funciones objetivo, o una combinaci\'on lineal de ambas.



\begin{center}
\includegraphics[width=1\textwidth]{imagenes/figura7}
\end{center}




\newpage
\section{Algoritmo Propuesto}\label{sec:algoritmo}
\subsection{Explicaci\'on}

\subsubsection{Colonia de hormigas}

El algoritmo implementado esta basado en la metaheur\'istica de \textbf{colonia de hormigas}.

La idea general es tener una matriz que representa a la \textbf{feromona}, esta matriz reprensenta la region de entrada, en otras palabras el valor en la posicion (x,y) de la matriz de feromona representa el valor de la feromona en dicha posicion del suelo. En un principio se inicaliza con todos los valores en 0, como se ve a continuacion:


\begin{center}
\includegraphics[width=7cm]{imagenes/feromona0}
\includegraphics[width=7cm]{imagenes/feromona01}
\end{center}

La matriz de fermona esta \textbf{discretizada} con un valor \textbf{configurable por par\'ametro}. Por ejemplo, si la regi\'on es de 100x100 metros, pero la discretizacion de la feromona la seteamos en 10 metros, vamos a tener una matriz de feromona de 10x10. Es clave notar en este punto que cuanto menor es el valor de la discretizaci\'on de la feromona mayor cantidad de puntos en la matriz y por lo tanto vamos a lograr mejores resultados pero a su vez resultados con tiempo computacional m\'as alto (ver secci\'on de experiemntaci\'on).

Por otro lado, contamos con una estructura auxiliar, una matriz de tamaños iguales que la matriz de feromona, esta estructura contiene 0 y 1 dependiendo de la \textbf{disponibilidad} del punto de la feromona. Esto es utilizado para saber cuando una feromona esta disponible, dado que los vamos a ir tapando con el correr de las iteraciones y por otro lado, en casos de que la regi\'on original no sea rectangular o este rotada, la matriz de feromona se arma de forma tal que la regi\'on quede incluida y los espacios fuera de la regi\'on se setean como \textbf{no disponibles} en esta nueva estructura. 

En la siguiente figura vemos un ejemplo donde la seccion amarrilla seria la regi\'on, y la azul seria la matriz de feromona, por lo tanto en la matriz de disponibilidad en los casilleros correspondientes a partes azules tendriamos un 1 indicando que no esta disponible esa feromona, ya que no esta dentro de la regi\'on.

\begin{center}
\includegraphics[width=6cm]{imagenes/ejemplo1}
\end{center}


Lo siguiente es, ejecutar una cantidad configurable de iteraciones el algoritmo obteniendo en cada soluci\'on, un conjunto de soluciones que van \textbf{actualizando} la matriz de feromona y en cada paso del algortimo vamos \textbf{chequeando y guardandonos la mejor soluci\'on}, siendo la mejor soluci\'on la que tenga el valor m\'as alto del ogip tapado. 

\newpage

En las siguientes imagenes podemos ver algunos ejemplos de feromonas luego de varias iteraciones.

\begin{center}

\includegraphics[width=7cm]{imagenes/fero0}
\includegraphics[width=7cm]{imagenes/fero1}
\end{center}


El algoritmo esta dividido en dos partes, la \textbf{iteraci\'on inicial} y el \textbf{resto de las iteraciones} (decisi\'on tomado para facilitar la implementaci\'on). 

A continuaci\'on se explica el trabajo de cada hormiga, es decir la forma de conseguir cada soluci\'on dependiendo si es la iteraci\'on inicial o el resto de las iteraciones.

\textbf{Iteraci\'on Inicial:} En esta iteraci\'on se crean una \textbf{cantidad configurable} de soluciones random, y para cada una de estas se actualiza la matriz de feromonas de la forma que corresponda. 

Para crear una soluci\'on random, la idea principal es meter pads centrados en puntos random de la regi\'on, teniendo en cuenta que sean v\'alidos (que esten dentro de la regi\'on, sin pisarse y sin interceptarce con una restrinci\'on) hasta no poder meter mas pads. Notar que tambien se elije la semilla de forma random. El problema en este caso es decidir cuando ya no se pueder meter pads (dado que estamos trabajando en un plano \textbf{continuo}), por lo tanto se considero tener un \textbf{valor configurable} de intentos de meter pad fallidos. En caso de fallar en insertar el pad random esa cantidad de veces entonces se considera que no entra ning\'un pad mas y se retorna la soluci\'on. 

A continuaci\'on podemos ver un ejemplo de una soluci\'on random, donde termino de poner pads porque se \textbf{creyo} que no entraban mas, pero podemos ver en la otra figura, marcado con amarrillo 3 posibles pads que se nota que no se encontraron.

\begin{center}
\includegraphics[width=7cm]{imagenes/ejemplo2}
\includegraphics[width=7cm]{imagenes/ejemplo3}
\end{center}




\textbf{Resto de las iteraciones:} En esta iteracio\'n se crean una \textbf{cantidad configurable} de soluciones no random, y para cada una de estas se actualiza la matriz de feromonas de la forma que corresponda.

Para crear una \textbf{soluci\'on no random}, la idea principal es agarrar el punto de la feromona m\'as \textbf{caliente} y generar una \textbf{cantidad configurable} de pads random que puedan tapar esa feromona. 

Nota: si no consigo ningun pad random que tape dicho punto de la feromona (dado que los pads pueden ser invalidos, por ejemplo tocando una restriccion) , descarto ese punto de la feromona. 


La manera de conseguir un pad random que tape un punto c, es tan simple como un pad posicionado en cualquier lugar de la regi\'on que tape a c. \textbf{Esto esta hecho de esta forma para que cada soluci\'on sea distinta de las otras.}

Una vez tapado el punto de la feromona, se acomoda el pad, se tapan el resto de los puntos de la feromona que este pad tapo (tambi\'en actualizamos la matriz de disponibilidad) y se itera hasta no tener m\'as puntos feromona para tapar. 

\textbf{Actualizaci\'on de la feromona:} Para actualizar la feromona utilizamos una funci\'on llamada \textbf{esBuenasoluci\'on} que determina si una soluci\'on es buena o mala. Contamos con un \textbf{par\'ametro configurable} dado que se tiene 3 formas distintas de ver si una soluci\'on es buena o mala

\begin{enumerate}
\item Opci\'on 0: Una soluci\'on es buena si el ogip cubierto por esta soluci\'on es mas que el 75\% del total del ogip, en otro caso es una mala soluci\'on.
\item Opci\'on 1: Una soluci\'on es buena si el ogip cubierto por esta soluci\'on es mas que la mitad de la suma del maximo y minimo ogip hasta el momento, en otro caso es mala soluci\'on.
\item Opci\'on 2: Una soluci\'on es buena si el ogip cubierto por esta soluci\'on es mas que el promedio de los ogip de las soluciones calculadas hasta el momento, en otro caso es mala soluci\'on. 
\end{enumerate}

En caso de que la soluci\'on sea buena se recorre la matriz de feromonas, aumentanto (calentando) en cada casillero (punto de la feromona) tapado por un pad en dicha soluci\'on un valor igual al ogip en ese punto (normalizado) por una constante configurada por parametro. Es analogo para el caso de una soluci\'on mala, solamente que disminuyendo (enfriando).


\textbf{Importante:} Tener en cuenta que para todos los casos, una vez elegido el pad que voy a agregar a mi soluci\'on, a este pad lo \textbf{acomodo} haciendo que se mueva para la direccion mas cercana a otro pad o borde, hasta chocarce con el. De esta forma obtengo soluciones con pad pegados entre si y no aparecen huecos.

Una vez terminadas todas las iteraciones vamos a tener guardado la mejor soluci\'on, el n\'umero de iteracion de donde sali\'o dicha soluci\'on y el tiempo en conseguirla.

Notar que la mejor soluci\'on no necesariamente es de la \'ultima iteraci\'on, es por eso que la vamos guardando en cada iteraci\'on y tambi\'en vamos guardando el n\'umero y tiempo de la iteraci\'on de donde se origin\'o la mejor soluci\'on.

\subsubsection{Colonia de hormigas Versi\'on Alternativa}

Tambi\'en se desarrollo un algoritmo extra, tambi\'en basado en \textbf{colonias de hormigas}, implementado de forma casi identica al anterior, salvo que en lugar de tener una \'unica matriz de feromonas, tenemos una matriz de feromonas por cada semilla, es decir si tengo 5 semillas (tipo de pad) tengo 5 matrices de fermona y una \'unica matriz de disponibilidad. 

Entonces a la hora de actualizar la matriz de feromona, actualizamos para cada matriz los puntos tapados por los pads que tienen esa semilla. Por ejemplo, si tenemos 2 semillas, una grande y una chica, en una matriz de feromona vamos a actualizar los puntos tapados por los pads chicos y en la otra matriz de feromona actualizamos los puntos tapados por pads grandes. 

A la hora de buscar la feromona m\'as caliente, se busca entre todas las matrices de feromona.

El objetivo de esto es notar la variaci\'on de la feromona, y tratar de ver que en algunos casos es conveniente poner los pads mas grandes en los puntos mas calientes mientras que los lugares restantes, por ejemplo a los costados de los lugares de alto valor, se agregan pads mas chicos. 

\newpage

A continuaci\'on se muestran ejemplos de feromonas para esta versi\'on alternativa. Podemos ver en la figura de la izquierda como se modific\'o la feromona en los lugares donde tiene los pads mas grandes. Por otro lado, la feromona de la derecha se nota como esta modificada en los lugares que rodean a los pads mas grandes, concluyendo que se pusieron pads mas chicos en los alrededores de los mas grandes (ver secci\'on experiementaci\'on)

\begin{center}

\includegraphics[width=7cm]{imagenes/ferov20}
\includegraphics[width=7cm]{imagenes/ferov21}
\end{center}

A continuaci\'on podemos ver algunos ejemplos de soluciones obtenidas sobre una misma instancia cambiando los par\'ametros (ver m\'as en secci\'on experimentaci\'on).

\begin{center}

\includegraphics[width=7cm]{imagenes/ejemplo7}
\includegraphics[width=7cm]{imagenes/ejemplo5}
\end{center}


\begin{center}
\includegraphics[width=7cm]{imagenes/ejemplo6}
\includegraphics[width=7cm]{imagenes/ejemplo4}

\end{center}

Notar que en los dos primeros casos no quedaron huecos mientras que en los \'ultimos dos si, esto se debe a la discretizaci\'on de la feromona (en los \'ultimos dos casos la discretizacion es un n\'umero m\'as alto, por lo tanto tenemos menos lugares que tapar, provocando huecos).

\newpage


\subsection{Pseudocodigo}
\subsubsection{Algoritmo principal}

\begin{verbatim}
    resolver() {
        inicializarFeromonas();
        ejecutarIteracionInicial();
        return ejecutarProximassoluciones();
    }
\end{verbatim}

\textbf{InicializarFeromonas()} es una funci\'on que inicializa la feromona como una matriz del tamaño de la regi\'on, en caso que la regi\'on no se rectangular la inicializa con el rect\'angulo mas chico que contenga a la regi\'on. 

Tambi\'en inicializa una matriz disponibilidad del mismo tamaño que la feromona pero esta contiene 1 o 0 dici\'endonos si una feromona es validad o no, sea porque ya esta usada o porque la regi\'on es m\'as chica que la matriz de feromonas y en esa feromona no tenemos regi\'on.


\begin{verbatim}
    ejecutarIteracionInicial() {
        soluciones = crearsolucionesRandom();
        for (soluci\'on soluci\'on : soluciones) {
            if(esBuenasoluci\'on()){
                actualizarFeromona(soluci\'on, OperacionFeromona.Calentar);
            } else {
                actualizarFeromona(soluci\'on, OperacionFeromona.Enfriar);
            }
        }
    }
\end{verbatim}		

\textbf{ejecutarIteracionInicial()} crea una cantidad seteada por par\'ametro de soluciones randoms y para soluci\'on chequea si es una buena o mala soluci\'on, la funcion \textbf{esBuenasoluci\'on()} cambia seg\'un un valor pasado por parametro, pero en definitiva, devuelve \textbf{true} si es una soluci\'on considerada buena o \textbf{false} si es considerada mala. 

En caso de que la soluci\'on sea buena, \textbf{calentamos} la matriz de feromonas y \textbf{enfriamos} en caso contrario.

La funci\'on \textbf{actualizarFeromona()} simplemente recorre la matriz \textbf{calentando} o \textbf{enfriendo} cada valor respectivamente. Pero notar que la calienta \textbf{teniendo en cuenta el valor del ogip en ese punto}, es decir, se normaliza el ogip y en cada punto de feromona se calienta o enfr\'ia un valor igual a $escalar * feromonaNormalizadaEnElPunto$. Esto es para que el algoritmo de colonias de hormigas tenga en cuenta los valores originales del problema para generar sus soluciones.


La funci\'on \textbf{esBuenasoluci\'on()}, tiene 3 opciones que se cambian dependiendo de un valor pasado por par\'ametro

\begin{enumerate}
\item Opci\'on 0: Una soluci\'on es buena si el ogip cubierto por esta soluci\'on es mas que el 75\% del total del ogip, en otro caso es una mala soluci\'on.
\item Opci\'on 1: Una soluci\'on es buena si el ogip cubierto por esta soluci\'on es mas que la mitad de la suma del m\'aximo y m\'inimo ogip hasta el momento, en otro caso es mala soluci\'on.
\item Opci\'on 2: Una soluci\'on es buena si el ogip cubierto por esta soluci\'on es m\'as que el promedio de los ogip de las soluciones calculadas hasta el momento, en otro caso es mala soluci\'on. 
\end{enumerate}

\newpage

\begin{verbatim}	
crearsolucionesRandom() {
    ret = new soluci\'on()
    while (hasta que el area deje de cambiar) {
        Coordenada c = generarCoordenadaRandom()
        Pad pad = crearPadConSemillaRandamCentradaEnCoordenada(c)
        if (padValido(pad)){
            Pad padAcomodado = acomodarPad(pad);
            agregarPadAsoluci\'on(ret,padAcomodado);
        }
    }
    return ret;
}
\end{verbatim}	

La condici\'on del while corta cuando ya no se pueden meter mas pads en mi soluci\'on random, esto se hace teniendo en cuenta una cantidad fijada por parametro de intentos de meter un pad, es decir, intento meter pads en la soluci\'on y si la cantidad de veces que no pude meter es mayor al par\'ametro seteado, se asume que no entran m\'as pads y sale del while. 

Esto se hace para tratar de manejar la regi\'on en un plano \textbf{continuo} y para tratar de solucionar el problema de saber cuando ya no entran mas pads.

\textbf{GenerarCoordenadaRandom()} generada un x,y random dentro de la region. 

\textbf{crearPadConSemillaRandamCentradaEnCoordenada(c)()} elije una semilla random y crea un pad centrado en c

\textbf{padValido(pad)} chequea si el pad no se pisa con ninguna restricci\'on, ni se va fuera de la regi\'on, ni se pisa con otro pad ya agregado a la soluci\'on.

\textbf{acomodarPad(pad)} mueve el pad para una direcci\'on random hasta chocarse son un borde u otro pad sin destapar el centro. Eso se hace para tratar de pegar todos los pads en la soluci\'on.

\textbf{agregarPadAsoluci\'on()} agrega el pad a la soluci\'on.

\textbf{ejecutarProximassoluciones()} ejecuta una cantidad de veces igual a \textbf{cantIteraciones}, un algoritmo similar a \textbf{crearsolucionesRandom()}, llamado \textbf{generarsolucionesMaximaTemperatura()}. De esta forma se crean soluciones durante muchas iteraciones.

\textbf{generarsolucionesMaximaTemperatura()} es exactamente igual a \textbf{crearsolucionesRandom()} nada mas que cambiando el m\'etodo \textbf{crearsolucionesRandom()} por \textbf{generarsolucionesMaximaFeromona()}. Esto significa que crea una cantidad configurable de soluciones de m\'axima feromona. Y para cada soluci\'on chequea si es buena o mala actualizando la feromona como corresponda al igual que lo hac\'iamos en la iteraci\'on inicial.

Una soluci\'on de m\'axima feromona es una soluci\'on que tiene en cuenta el valor de la fermona para generarse y se genera con el siguiente algoritmo:


\begin{verbatim}
construirsoluci\'onMaximaTemperatura() {
    sol = new soluci\'on();
    while (mientra que tenga feromonas disponibles) {
        Feromona p = getMaximaFeromona()
        if (es una feromona que se puede tapar) {
            for (int j = 0; j < getCantIntentosTaparFeromona(); j++) {
                nuevoPad = generarNuevoPad(p);
                if (esPadValido(nuevoPad)) {
                    nuevoPad = acomodarPad(nuevoPad);
                    agregarPadAsoluci\'on(nuevoPad);
                    break;
                }
            }
        }
    }
    return sol;
}
\end{verbatim}

Mientras tenga feromonas disponible, es decir que todavia no las tapas (y est\'an dentro de la regi\'on) ejecuto todo el c\'odigo dentro del while. 

Obtengo la m\'axima feromona con \textbf{getMaximaFeromona()} y chequeo si es una posible feromona a tapar, dado que podr\'ia pasar que esa feromona este en una restricci\'on. 
Una vez que ya se que esa feromona la puedo tapar, trato de generar \textbf{getCantIntentosTaparFeromona()} pads (este valor es seteado por parametro). 

La idea general es que para cada iteracion genero un pad random que tape a la feromona, chequeo si es valido, y si es valido lo agrego a la soluci\'on y dejo de intentar tapar esta feromona.

Si no encontre ningun pad que sea valido y tape a la feromona en \textbf{getCantIntentosTaparFeromona()} intentos entonces ya esa feromona la descarto. 

\textbf{Notar que a los pads los acomodo (al igual que antes) para que queden pegados a otros pads o al borde.}

\textbf{Nota importante:} Tanto en la generaci\'on de soluciones random o las soluciones de m\'axima feromona, a la hora de fijarse si es una buena o mala soluci\'on para actualizar la feromona, se chequea si es la \textbf{mejor soluci\'on}, en caso de ser la mejor se \textbf{guarda}. Esto es para guardar la mejor soluci\'on en el camino, podr\'ia llegar a pasar que la mejor soluci\'on la encuentre en iteraciones iniciales y las siguientes sean peores.


\subsubsection{Alternativa: Muchas Feromonas}
A parte de tener un algoritmo de colonias de hormigas que ejecuta con una \'unica feromona, se program\'o una versi\'on alternativa donde contamos con mas de una feromona. Es decir para cada soluci\'on, en lugar de tener una unica feromona, \textbf{tenemos una matriz de feromona para cada semilla}.
 
Para esto se modific\'o leventente el c\'odigo teniendo en cuenta que ahora manejamos arrays de feromonas, y cambian levemente los algoritmos de actualizar la feromona. En particular, el algoritmo para obtener c\'ual es la m\'axima feromona es el siguiente:


\begin{verbatim}
getMaximaFeromona() {
    result = null;
    for (f in todas las feromonas) {
        if(si la feromona f tiene valores dissponibles)
            if(maximo de f > result)
                result = f1;
    }
    return result;
}
\end{verbatim}

El algoritmo es bastante sencillo, la idea es recorrer todas las feromonas buscando el m\'aximo valor. 

Tener en cuenta que a la hora de actualizar la feromona, cada feromona solo se actualiza donde corresponde. Por ejemplo,  la feromona correspondiente a la semilla 0 se calienta o enfr\'ia solo en los puntos donde la soluci\'on puso semillas 0. 
\newpage
\section{Experimentación}\label{sec:experimentacion}

En esta seccion se presentaran los resultados de la experimentacion realizada. Pero previamente es necesario conocer los algoritmos con los cuales se comparo nuestra metaheuristca.

\begin{enumerate}
\item Scip: Este algoritmo esta realizado con programacion lineal entera y se explica en detalle en la seccion \textbf{Programacion Lineal Entera}.
\item Goloso: Este algoritmo es un simple Goloso, es decir, en cada iteracion agrega a la solucion en pad que mas ogip tape.
\item Goloso Maximos Locales: Este algoritmo, en cada iteracion busca el pad de mayor ogip (al igual que el goloso), pero como esto esta ligado a la discretizacion, a este pad se lo mueve tratando de ubicarlo en algun lugar cercano donde tenga mayor ogip, es decir, no importa al 100\% la discretizacion, dado que se consigue un pad centrado en la discretizacion (este pad es el pad con mayor ogip de todos los pads centrados en la discretizacion) y luego busco un maximo local en los alrededores y una vez encontrado lo agrega a la solucion.
Para esto se utiliza un parametro \textbf{Paso Movimiento Pad} que indica el paso que se tiene en cuenta a la hora de mover el pad buscando el maximo local. 
Una vez que no tengo mas pads disponibles puede haber pasado que al mover los pads, no tenga mas pads disponibles de los centrados en la discretizacion, pero si existen huecos donde entran otros, por lo tanto, se \textbf{re-discretiza} el area no tapada hasta el momento, se hace una discretizacion mas fina, y para esto se usa el parametro \textbf{Paso mejora Discretizacion} que indica en cuanto se achica la discretizacion. Luego se vuelve a calcular los posibles pads para esta nueva discretizacion. Notar que solo se discretiza mas fino los sectores no tapados por los pads provenientes de la discretizacion mas gruesa.
\end{enumerate}
\newpage
\section{Conclusión}\label{sec:conclusion}

Como se pudo ver en la seccion de experimentacion, los resultados de Colonia de Hormigas (en cualquiera de sus dos versiones) no siempre son mejores que los de Goloso y Goloso Maximos Locales. Y vale aclarar que nunca son mejores que los de Scip, pero recordemos que Scip acepta superposiciones, por lo que esto nos hace estar en desventaja. 
Pero existen muchisimos casos en los que nuestro algoritmo es mejor ue Goloso y Goloso de Maximos Locales.

Luego de hacer un analisis (ver seccion experimentacion) con respecto al tiempo de ejecucicon, podemos concluir que a medida que se aumentan la cantidad de soluciones por iteraciones, el tiempo aumenta, lo cual es un resultado esperado. En algunos casos aumenta mas y en otros menos, pero en general siempre aumenta.
En cambio si aumentamos la cantidad de iteraciones no siempre aumenta el tiempo, esto es debido a que el tiempo calculo es el tiempo que se tardo en encontrar la mejor solucion por lo tanto, si la solucion se encuentra en una iteracion temprana, el tiempo va a ser menor que si se encontro en una iteracion mas lejana. Esto es independiente de si tiene mas o menos iteraciones. Notar que si tiene mas iteraciones aumenta la probabilidad de encontrar una mejor solucion a la solucion parcial encontrada hasta el momento. 
Se puede ver que en algunas corridas, la mejor solucion se encontro siempre en la ultima iteracion, dando esto un indicio de que si se correrian mas iteraciones se podrian encontrar mejores soluciones. Lamentablemente el tiempo computacional no nos permitio probar con muchas mas iteraciones pero queda como trabajo futuro tratar de hacer este analisis.

Se puede ver facilmente que la variacion de la discretizacion es clave en el tiempo, cuanto mas discretizamos mas tiempo se tarda.

En comparacion con los otros algoritmos, nuestro algoritmo siempre es mas lento, pero se pueden ver muchisimos casos donde aunque no se encuentre la mejor solucion, se consiguen mejores tiempos, es decir existen soluciones que no son las mejores pero que el tiempo es mucho menor comparado con tiempos de los otros algortimos. Tambien existen muchos casos donde se puede conseguir mejores resultados pero en mas tiempo. Igualmente se puede ver que en muchos casos se pueden conseguir soluciones mejores que las soluciones de Goloso y Goloso Maximos Locales y en menos tiempo. 

Con esto podemos decir que nuestro algoritmo consigue, en algunos casos, soluciones mejores (tanto en tiempo como en ogip) y en algunos casos no tan buenos resultados. 

En general nuestro algoritmo es mejor en soluciones no tan discretizadas, es decir, cuanto mas discretizamos mejoran todos los algortimos, pero en los otros algoritmo la mejora es levemente mas grande, pero esto no es un problema, dado que las instancias de prueba que corrimos son instancias chicas, por lo tanto, si uno quisiera correr un caso real, donde la instancia tiene muchisimos metros de area, ninguno de los algoritmos podria llegar a discretizar muy profundo, por lo tanto nuestro algoritmo, podria llegar a conseguir mejores soluciones (en tiempo y ogip) que los otros algoritmos.

El ogip aumenta al incrementar la cantidad de soluciones por iteracion y tambien al incrementar las iteraciones. Pero notar que existen casos donde aunque uno aumenta la cantidad de iteraciones, el algortimo \textbf{se plancha} y no consigue nada mejor. Esto es porque la mejor solucion la consiguio de forma temprana (no quita que exista la posibilidad de que aparezca una mejor en iteraciones mas avanzadas).

Con respecto a los modos de chequear que una solucion es buena o no, en este trabajo se probaron 3 distintos, y se pudieron notar variaciones entre ellas, pero queda a trabajo futuro investigar que otras opciones de chequeo de buenas soluciones se pueden utilizar.

Finalizando, podemos concluir que Colonia de Hormigas nunca le gana a Scip (con repeticiones), en muchos casos podemos obtener mejores soluciones que los demas algoritmos y en muchos casos podriamos obtener soluciones mas lentas. Por lo tanto a la hora de decidir que algoritmo utilizar deberia hacerce un analisis de lo que uno prefiere, si priorizar el resultado o priorizar el tiempo, o buscar un intermedio, etc. 

En general, priorizando un \textbf{intermedio}, nuestro algoritmo tiene todas las chances de ganarle a los demas.

\end{document}
