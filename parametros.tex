\newpage
\section{Parametros}

En esta seccion se explicaran los parametros utilizados a la hora de hacer la experimentacion de todos los algoritmos. 

Notar que los algoritmos utilizados para comparar resultados son:

\begin{enumerate}
\item Scip (S)
\item Goloso (G)
\item Goloso Maximos Locales (GML)
\end{enumerate}

todos estos seran descriptos con mas detalles en la seccion \textbf{experimentacion}

Los parametros usados para estos algoritmos son:


\begin{enumerate}
\item Nx: Valor de la discretizacion del eje X.
\item Ny: Valor de la discretizacion del eje Y.
\item PMD: \textbf{Paso Mejora Discretizacion}, en el algoritmo GML se utiliza para \textbf{re-discretizar}, leer la explicacion en la seccion \textbf{experimentacion}
\item PMP: \textbf{Paso Movimiento Pad}, en el algoritmo GML se utiliza para el paso en que se mueven los Pads, a la hora de buscar el maximo local, leer la explicacion en la seccion \textbf{experimentacion}
\end{enumerate}

Por otro lado, para el algoritmo de colonia de hormgias se utilizaron los siguientes parametros:
							
\begin{enumerate}
\item IMPSR: \textbf{Intentos Meter Pad Solucion Random}, para ver cuando se termina de intentar meter pad en las soluciones tipo random (recordar que trabajamos en el continuo, y tenemos que decididr cuando ya creemos que se lleno la region).
\item CITF:  \textbf{Cantidad Intentos de Tapar una Feromona}, cantidad de pads que tapan a una feromona, para cada uno pruebo si es valido. Si ninguno es, se descarta esa feromona.
\item CSRI: \textbf{Cantidad Soluciones Random Iniciales}, cantidad de soluciones de la primera iteracion. (Soluciones random)
\item CI: \textbf{Cantidad Iteraciones}, cantidad iteraciones luego de la inicial.
\item CSNRPI: \textbf{Cantidad Soluciones No Random Por Iteracion}, cantidad de soluciones por iteracion (Luego de la inicial).
\item MCBS: \textbf{Modo Chequeo Buena Solucion}, el modo para chequear cuando una solucion es buena o mala (para enfriar o calentar la feromona, explicado mejor en la seccion \textbf{Algoritmo Propuesto}).
\item DF: \textbf{Discretizacion Feromona}, la discretizacion de la matrix de feromonas.
\item FCF: \textbf{Factor Cambio Feromona}, factor que se multiplica al actualizar la feromona (tambien se lo multiplica por el ogip normalizado)
\end{enumerate}












